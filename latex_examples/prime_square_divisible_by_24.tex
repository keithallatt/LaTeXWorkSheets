%! Author = kallatt
%! Date = 2022-04-06

% Standalone file on its own, but using special

% Preamble
\documentclass[11pt]{article}

% Package
\usepackage{amsmath}
\usepackage{amssymb}
\usepackage{amsthm}
\newtheorem{theorem}{Theorem}[section]
\renewcommand\qedsymbol{$\blacksquare$}

\linespread{1.5}

% Document
\begin{document}

\begin{theorem}
    % TEX THEOREM START %
    For any prime $p \geq 5$, $p^2$ is one more than a multiple of 24.
    % TEX THEOREM END %
\end{theorem}

\begin{proof}
    % TEX PROOF START %
    Let $p$ be a prime, $p \geq 5$.

    The quantity ${p^2 - 1}$ equals ${(p-1)(p+1)}$, and as ${p \geq 5}$ prime, ${3 \not|\; p}$.
    If ${p \equiv 1 \pmod{3}}$, then ${p-1 \equiv 0 \pmod{3}}$, and thus ${3 \mid p-1}$.
    Otherwise, ${p \equiv 2 \pmod{3}}$, and so ${p+1 \equiv 0 \pmod{3}}$, and thus ${3 \mid p+1}$.
    Thus ${(p-1)(p+1)}$ is a multiple of 3 for all ${p \geq 5}$ prime.

    As $p$ is prime, and ${2 \not|\; p}$, ${p-1}$ and ${p+1}$ are both even.
    As $\frac{p-1}{2}$, and $\frac{p+1}{2}$ are consecutive integers, one must be even, and thus one of ${p-1}$ and ${p+1}$ is a multiple of 4.
    As one of ${p-1}$ and ${p+1}$ is a multiple of 4, and the other is certainly even, ${(p-1)(p+1)}$ is a multiple of 8.

    As ${(p-1)(p+1)}$ is a multiple of 3 and a multiple of 8, and $\gcd(3, 8) = 1$, ${(p-1)(p+1)}$ must be a multiple of 24.

    Thus ${p^2-1}$ is a multiple of 24, and $p^2$ is one more than a multiple of 24.

    % TEX PROOF END %
\end{proof}

\end{document}